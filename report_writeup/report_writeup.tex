

\documentclass[11pt,notitlepage]{article}


%\documentclass[twocolumn,secnumarabic,amssymb, nobibnotes, aps, prd]{revtex4-1}
%\documentclass[aps,preprint]{revtex4}
%\documentclass[pra,preprint]{revtex4}
%\documentclass[pra,twocolumn]{revtex4}

\usepackage[lmargin=1in,rmargin=1in,tmargin=1in,bmargin=1in]{geometry}
\usepackage{setspace} %\doublespacing
\usepackage[pdftex]{graphicx}
\usepackage{subfig}
\usepackage{latexsym}
\usepackage{amsfonts}
\usepackage{amsmath,revsymb,graphicx} 
\usepackage{natbib}
\usepackage{longtable}
\usepackage{titling}
\usepackage[
	pdfauthor={Brian Weinstein},
	pdftitle={The Association Between Felonies in NYC and Weather and Temporal Conditions},
	bookmarks=true,
	colorlinks=true,
	linkcolor=blue,
	urlcolor=blue,
	citecolor=blue,
	pdftex,
	linktocpage=true
	]{hyperref}
\usepackage[textsize=tiny]{todonotes}
\usepackage{authblk}
\usepackage{float}
\usepackage{caption}
\usepackage{xspace}
\usepackage{underscore} % can't use underscores in file names (e.g., image names) when using this package
%\usepackage{enumitem} \setlist{nosep}


\usepackage{xcolor}
\usepackage{adjustbox}
\usepackage{verbatim}
\definecolor{shadecolor}{rgb}{.9, .9, .9}
\newenvironment{codeSmall}%
   {\par\noindent\adjustbox{margin=1ex,bgcolor=shadecolor,margin=0ex \medskipamount}\bgroup\minipage\linewidth\verbatim\footnotesize}%
   {\endverbatim\endminipage\egroup}


\newcommand{\degf}{^\circ\text{F}}


\usepackage[compact]{titlesec}
%\titlespacing{\section}{0pt}{4pt}{3pt}
%\titlespacing{\subsection}{0pt}{3pt}{2pt}

\titleformat*{\section}{\Large\bfseries}
\titleformat*{\subsection}{\large\bfseries}



\begin{document}

\title{The Association Between Felonies in NYC and Weather and Temporal Conditions}
\author{Brian Weinstein}
\affil{\textit{Columbia University} \\ \textit{STAT W4201: Advanced Data Analysis}}
\date{May 2, 2016}

\maketitle



\begin{abstract}


\noindent \textbf{Background:} The New York City Police Department recently released incident-level felony data to the NYC Open Data portal. The dataset includes timestamped information for felonies committed in NYC.

We first examine the association between the daily number of felonies committed in NYC in 2015 and: temperature, presence of precipitation, day of week, federal and New York holidays, and school days. Second, we examine the association between day-to-day changes in the number of felonies and large increases in temperature ($>8 \degf$ from the previous day).

\noindent \textbf{Methods and Results:} Using simple linear regression, the data provides overwhelming evidence that felonies is associated with temperature. For every $1 \degf$ increase in temperature, there are, on average, 1.38 additional felonies per day (95\% CI 1.23 to 1.53 felonies, two sided p-value $<2\times10^{-16}$).

Using multiple linear regression, after accounting for temperature, there is convincing evidence that the number of felonies on a given day is associated with precipitation. On average there are 21 fewer felonies on days with precipitation than on days without (95\% CI 4 to 38 fewer felonies; two-sided p-value 0.014990). The data also provides overwhelming evidence that felonies is associated with day of week (p-value $3 \times 10^{-6}$ from an extra sum of squares F-test).

Initially, it did not appear that the number of felonies was associated with holidays or with school days, but after removing 17 observations with high studentized residuals, these variables became significant. From the revised model, there is suggestive, but inconclusive evidence that felonies is associated with holidays --- on average, holidays have 13 fewer felonies than non-holidays (95\% CI 0 to 27 fewer felonies; two-sided p-value 0.05275). There is moderate evidence that felonies is associated with school days --- on average, school days have 7 more felonies that non-school days (95\% CI 1 to 13 more felonies; two-sided p-value 0.03306).

After accounting for day of week, the data provides no evidence that day-to-day changes in the number of felonies is associated with large increases in temperature (two-sided p-value 0.10006).

\noindent \textbf{Conclusions:} There is a clear association between warmer temperatures and an increased number of felonies. Presence of precipitation, day of week, holidays, and school days are also related to the number of felonies that occur. There is no evidence that the day-to-day changes in the number of felonies is associated with large increases in temperature.




\end{abstract}


\pagebreak


\section{Introduction}



After many years of pressure, the New York City Police Department (NYPD) recently released incident-level felony data to the NYC Open Data portal as part of their initiative to improve their accessibility, transparency, and accountability. Prior to this release, felony data had only been provided in an aggregated format (by week and police precinct), and was done so only in PDF and Excel files on a weekly and quarterly basis.

In this paper, we use the newly-released data to examine the association between the daily number of felonies committed in New York City (NYC) and: outside air temperature, precipitation, day of week, federal and New York (NY) holidays, and public school days.

\subsection{Questions of Interest}

In this paper we study three main questions of interest:

\begin{enumerate}
\setlength\itemsep{-3pt}
\vspace*{-3mm}

\item Are felonies associated with temperature? After taking temperature into account, is felonies associated with precipitation, school days, holidays, and day of week?
\item Although there's no causal relationship, for a given set of these conditions, how many felonies can the NYPD reasonably expect?
\item After taking into account temperature, precipitation, school days, holidays, and day of week; are the day-to-day changes in the number of felonies associated with large ($>8 \degf$) increases in temperature?
\end{enumerate}



\subsection{Dataset}
\label{sec:dataset}

The dataset contains 365 observations, one for each date in 2015. The class, description, and source for each variable in the dataset is outlined below. Only those variables used/referenced in the analyses are included here --- redundant and untranformed variables that were removed during exploratory analysis are not described below.

\begin{itemize}
\setlength\itemsep{-3pt}
\vspace*{-3mm}
	
\item \texttt{felonies} (integer) is a count of the number of felonies committed on each day in NYC in 2015. The values are derived counts from the ``NYPD 7 Major Felony Incidents'' dataset in the NYC Open Data Portal [\href{https://data.cityofnewyork.us/Public-Safety/NYPD-7-Major-Felony-Incidents/hyij-8hr7}{data.cityofnewyork.us/d/hyij-8hr7}]. The felonies included in the dataset that contribute to the overall daily count are burglary, felony assault, grand larceny, grand larceny of motor vehicle, murder and non-negligent manslaughter, rape, and robbery.

\item \texttt{temp_min_degF} (numeric) is the minimum daily temperature on the given date (in degrees Fahrenheit), as reported by the New York Central Park Belvedere Tower weather station. The data was requested via the National Centers for Environmental Information [\href{http://www.ncdc.noaa.gov/cdo-web/search}{ncdc.noaa.gov/cdo-web/search}].


\item \texttt{any_precip} (factor) is an indicator variable, taking value ``1'' if there was any precipitation on the given date, and ``0'' otherwise. See \texttt{temp_min_degF} for source information.


\item \texttt{is_holiday} (factor) is an indicator variable, taking value ``1'' if the given date is a NY or federal holiday, or value ``0'' otherwise. The NY and federal holidays were defined using the lists provided by the NY State Department of Civil Service [\href{https://www.cs.ny.gov/attendance_leave/2015_legal_holidays.cfm}{cs.ny.gov/attendance_leave/2015_legal_holidays.cfm}] and U.S. Office of Personnel Management [\href{https://www.opm.gov/policy-data-oversight/snow-dismissal-procedures/federal-holidays/\#url=2015}{opm.gov/policy-data-oversight/snow-dismissal-procedures/federal-holidays/\#url=2015}], respectively.


\item \texttt{is_school_day} (factor) is an indicator variable, taking value ``1'' if NYC Public Schools were open and in session on the given date, or value ``0'' otherwise. Although the NYC Department of Education publishes this data to the NYC Open Data Portal, the historical data is only retained there for the current school year. Instead we scrape the attendance data from XML files in Aaron Schumacher's ``NYCattends'' Github repository [\href{https://github.com/ajschumacher/NYCattends/tree/master/xml}{github.com/ajschumacher/NYCattends/tree/master/xml}].
% link in the NYC Open Data Portal: \href{https://data.cityofnewyork.us/Education/Attendance-4PM-Report/madj-gkhr}{data.cityofnewyork.us/d/madj-gkhr}


\item \texttt{day_of_week} (factor) is a categorical variable indicating the day of week (Sunday=``1'', Monday=``2'' $, \ldots, $ Saturday=``7''). 

\item \texttt{felonies_diff} (numeric) indicates for a given date the difference in the number of felonies as compared to the previous day. On Jan. 3, 2015, for example, \texttt{felonies_diff} = 6, since there were 6 more felonies committed on Jan. 3 than on Jan. 2.

\item \texttt{temp_min_degF_diff} (numeric) indicates for a given date the difference in \texttt{temp_min_degF} as compared to the previous day. On Jan. 3, 2015, for example, \texttt{temp_min_degF_diff} = -1.98, since the daily minimum temperature was $1.98 \degf$ lower on Jan. 3 than on Jan. 2.

\item \texttt{temp_jump} (factor) is an indicator variable, taking value ``1'' if \texttt{temp_min_degF_diff} $> 8$, or value ``0'' otherwise. An increase of $> 8 \degf$ puts the day of interest in the top 10\% of day-to-day temperature increases in 2015.

\end{itemize}



\subsection{Report Overview}

In Section \ref{sec:eda} we briefly present some exploratory analysis and data transformations.

Then in Section \ref{sec:modelingFelonies} we model the daily number of felonies. Assumptions are discussed in Section \ref{sec:feloniesAssumptions}; an exploratory model is presented in Section \ref{sec:feloniesExploratoryModel}; then the analysis and model checking and improvement are presented in Sections \ref{sec:modelFeloniesMultipleRegression} and \ref{sec:modelFeloniesModelCheckingImprovement}.

In Section \ref{sec:modelingFeloniesDiff} we model the day-to-day changes in the number of felonies; again discussing our assumptions, analysis, and model checking and improvement in Sections \ref{sec:feloniesDiffAssumptions}, \ref{sec:modelFeloniesDiffMultipleRegression}, and \ref{sec:modelFeloniesDiffModelCheckingImprovement}.

Lastly, in Section \ref{sec:conclusions} we discuss our statistical conclusions and the scope of inference.





\section{Exploratory Analysis and Data Cleaning}
\label{sec:eda}

We first examine pairwise scatterplots of some of the numeric variables in the raw dataset, as shown in Figure \ref{fig:pairsNumericExclAcc}. From this figure we first notice that there are approximately linear relationships between \texttt{felonies} and \texttt{temp_min_degF}, and between \texttt{felonies} and \texttt{temp_max_degF}. There is strong collinearity between \texttt{temp_min_degF} and \texttt{temp_max_degF} (correlation: 0.969), however, so we remove one of these variables (\texttt{temp_max_degF}) from the covariates that will be used in the regression models.

\begin{figure}[!h]
  \vspace*{-3mm}
	\centering
	\captionsetup{width=0.9\textwidth}
	\includegraphics[width=6in]{figures/pairsNumericExclAcc.png}
	\vspace*{-3mm}
	\caption{Pairwise scatterplots of some of the numeric variables in the raw dataset.}
	\label{fig:pairsNumericExclAcc}
	\vspace*{-3mm}
\end{figure}


Figure \ref{fig:pairsNumericExclAcc} also shows that there is no linear relationship between \texttt{felonies} and \texttt{school_attendance_pct} (the percent of students present in school on a given day). Any non-school-day has 0\% attendance, so instead of using this as a numeric variable, we convert it to the \texttt{is_school_day} indicator variable, taking value ``1'' if NYC Public Schools were open and in session on the given date (i.e., if \texttt{school_attendance_pct} is $>0$), or value ``0'' otherwise.

Faceted boxplots of the categorical variables are shown in Figure \ref{fig:facetCategorical}.

\begin{figure}[!h]
  %\vspace*{-3mm}
	\centering
	\captionsetup{width=0.9\textwidth}
	\includegraphics[width=6in]{figures/facetCategorical.png}
	\vspace*{-5mm}
	\caption{Faceted boxplots of the categorical variables in the dataset.}
	\label{fig:facetCategorical}
	\vspace*{-3mm}
\end{figure}

\section{Modeling the Number of Felonies Per Day}
\label{sec:modelingFelonies}

In this section we use simple and multiple linear regression to model the number of felonies per day, addressing the first and second questions of interest.

\subsection{Assumptions}
\label{sec:feloniesAssumptions}

We first assume that each occurrence of a felony is an independent Bernoulli event with very low probability $p$. The sum of these Bernoulli events is the number of felonies that occur on a given day. This sum follows a binomial distribution where $n$ is the number of opportunities for a felony to occur --- as a rough approximation this might be on the order of the population of NYC ($\sim 8.4$ million). Since $n$ is large enough, we can approximate this binomial distribution with a normal distribution.

We also assume the four assumptions of linear regression:
\begin{itemize}
\setlength\itemsep{-3pt}
\vspace*{-3mm}

\item Linearity: \texttt{felonies} can be expressed as linear combination of the independent variables
\item Homoscedasticity (constant variance): $\text{Var}(Y|X_1,\ldots, X_p)$ is the same at all values of $X_1,\ldots, X_p$
\item Normality: residuals in the fitted model are normally distributed
\item Independence: residuals in the fitted model are independent
\end{itemize}


\subsection{Exploratory Model}
\label{sec:feloniesExploratoryModel}

As an exploratory step, we initially perform a simple linear regression of felonies on temperature. The regression summary is shown in Table \ref{tab:lm1}.

% lm1
%\begin{codeSmall}
%               Estimate Std. Error t value Pr(>|t|)    
%(Intercept)   213.11913    4.07082   52.35   <2e-16 ***
%temp_min_degF   1.38097    0.07709   17.91   <2e-16 ***
%\end{codeSmall}

% lm1
% xtable(as.data.frame(summary(lm1)$coefficients), digits=c(100, 2, 2, 2, 4), display=c("s", "f", "f", "f", "g"))

\begin{table}[ht]
\vspace*{-1mm}
\footnotesize
\centering
\begin{tabular}{rrrrr}
  \hline
 & Estimate & Std. Error & t value & Pr($>|t|$) \\ 
  \hline
(Intercept) & 213.12 & 4.07 & 52.35 & $<2 \times 10^{-16}$ \\ 
  temp\_min\_degF & 1.38 & 0.08 & 17.91 & $<2 \times 10^{-16}$ \\ 
   \hline
\end{tabular}
\captionsetup{width=0.9\textwidth}
\vspace*{-2mm}
\caption{Regression summary from the simple linear regression of \texttt{felonies}.}
\label{tab:lm1}   
\vspace*{-3mm}
\end{table}


%Residual standard error: 27.48 on 363 degrees of freedom
%Multiple R-squared:  0.4692,	Adjusted R-squared:  0.4678 
%F-statistic: 320.9 on 1 and 363 DF,  p-value: < 2.2e-16

There is overwhelming evidence of an association between felonies and temperature. For every $1 \degf$ increase in temperature, there are 1.38 additional felonies per day (95\% CI 1.23 to 1.53, two sided p-value $<2\times10^{-16}$).

What's most interesting here, however, are the observations with high residuals, as shown in the residual plot in Figure \ref{fig:lm1Residuals}. Some days with large residuals aren't modeled well by temperature alone --- Jan. 1, 2015, for example, a federal holiday, had many more felonies than we'd expect given the temperature. Including other covariates in the model, like \texttt{is_holiday} (an indicator as to whether the day is a NY/federal holiday), might help to account for some of this behavior. We next add these additional variables in Section \ref{sec:modelFeloniesMultipleRegression}.

\begin{figure}[!h]
  \vspace*{-3mm}
	\centering
	\captionsetup{width=0.9\textwidth}
	\includegraphics[width=4.25in]{figures/lm1Residuals.png}
	\vspace*{-3mm}
	\caption{Residual plot for the regression of \texttt{felonies} on \texttt{temp_min_degF}.}
	\label{fig:lm1Residuals}
	\vspace*{-3mm}
\end{figure}


There are some potentially problematic observations with high leverage or large studentized residuals, but there was no change in interpretation after removing these observations from the dataset. Also note that we tested higher order terms of \texttt{temp_min_degF} in a multiple regression, but only the first-order term was significant.


\subsection{Statistical Analysis}
\label{sec:modelFeloniesMultipleRegression}

We next incorporate additional covariates, performing a multiple linear regression of felonies on the variables shown in the regression summary in Table \ref{tab:lm4}.


% lm4
%\begin{codeSmall}
%                           Estimate Std. Error t value Pr(>|t|)    
%(Intercept)               210.71208    5.66649  37.186  < 2e-16 ***
%temp_min_degF               1.38099    0.08768  15.750  < 2e-16 ***
%any_precip1               -21.01037    8.59459  -2.445 0.014990 *  
%is_holiday1                -6.00627    7.68276  -0.782 0.434865    
%is_school_day1              5.94836    3.74662   1.588 0.113258    
%day_of_week2                0.06874    5.73913   0.012 0.990450    
%day_of_week3               -3.29691    5.70419  -0.578 0.563647    
%day_of_week4               -3.46950    5.71773  -0.607 0.544376    
%day_of_week5               -0.17653    5.68349  -0.031 0.975239    
%day_of_week6               19.01055    5.69764   3.337 0.000938 ***
%day_of_week7               14.21319    5.02454   2.829 0.004940 ** 
%temp_min_degF:any_precip1   0.15666    0.16512   0.949 0.343385    
%\end{codeSmall}


% lm4
% xtable(as.data.frame(summary(lm4)$coefficients), digits=c(100, 2, 2, 2, 4), display=c("s", "f", "f", "f", "g"))
\begin{table}[ht]
\vspace*{-1mm}
\footnotesize
\centering
\begin{tabular}{rrrrr}
  \hline
 & Estimate & Std. Error & t value & Pr($>|t|$) \\ 
  \hline
(Intercept) & 210.71 & 5.67 & 37.19 & $<2 \times 10^{-16}$ \\ 
  temp\_min\_degF & 1.38 & 0.09 & 15.75 & $<2 \times 10^{-16}$ \\ 
  any\_precip1 & -21.01 & 8.59 & -2.44 & 0.01499 \\ 
  is\_holiday1 & -6.01 & 7.68 & -0.78 & 0.4349 \\ 
  is\_school\_day1 & 5.95 & 3.75 & 1.59 & 0.1133 \\ 
  day\_of\_week2 & 0.07 & 5.74 & 0.01 & 0.9905 \\ 
  day\_of\_week3 & -3.30 & 5.70 & -0.58 & 0.5636 \\ 
  day\_of\_week4 & -3.47 & 5.72 & -0.61 & 0.5444 \\ 
  day\_of\_week5 & -0.18 & 5.68 & -0.03 & 0.9752 \\ 
  day\_of\_week6 & 19.01 & 5.70 & 3.34 & 0.0009384 \\ 
  day\_of\_week7 & 14.21 & 5.02 & 2.83 & 0.00494 \\ 
  temp\_min\_degF:any\_precip1 & 0.16 & 0.17 & 0.95 & 0.3434 \\ 
   \hline
\end{tabular}
\captionsetup{width=0.9\textwidth}
\vspace*{-2mm}
\caption{Regression summary from the multiple linear regression of \texttt{felonies}.}
\label{tab:lm4}   
\vspace*{-3mm}
\end{table}



%Residual standard error: 25.58 on 353 degrees of freedom
%Multiple R-squared:  0.5527,	Adjusted R-squared:  0.5388 
%F-statistic: 39.66 on 11 and 353 DF,  p-value: < 2.2e-16





After accounting for temperature, there is convincing evidence that felonies is associated with precipitation. On average there are 21 fewer felonies on days with precipitation than on days without (95\% CI 4 to 38 fewer felonies; two-sided p-value 0.014990). The data also provides overwhelming evidence that felonies is associated with day of week (p-value $3 \times 10^{-6}$ from an extra sum of squares F-test) --- compared to Sundays, Fridays have 19 more felonies on average (95\% CI 8 to 30 more felonies; two-sided p-value 0.000938), and Saturdays have 14 more felonies on average (95\% CI 4 to 24 more felonies; two-sided p-value 0.004940).

After accounting for temperature; the holiday indicator, the school day indicator, and the interaction between temperature and precipitation are not significant (two-sided p-values: 0.434865, 0.113258, and 0.343385, respectively).

\subsection{Model Checking and Improvement}
\label{sec:modelFeloniesModelCheckingImprovement}

To check the validity of our model, we examine a residual plot and Q-Q plot in Figure \ref{fig:lm4ResidualsQQ}. The residual plot doesn't reveal any significant violations of the linearity, constant variance, or independence assumptions,; and the Q-Q plot shows that the residuals aren't perfectly normal, but that normality isn't a bad approximation.

% two-column figure
\begin{figure}[!h]
  \vspace*{-3mm}
  \centering
  \captionsetup{width=0.9\textwidth}
  \subfloat%[Residual plot.]
  		{\includegraphics[width=0.46\textwidth]
  		{figures/lm4Residuals.png}\label{fig:lm4Residuals}}
  \hfill
  \subfloat%[Q-Q plot.]
  		{\includegraphics[width=0.46\textwidth]
  		{figures/lm4QQ.png}\label{fig:lm4QQ}}
  \vspace*{-3mm}
  \caption{Residual plot and Q-Q plot for the regression of \texttt{felonies} on the variables shown in Table \ref{tab:lm4}.}
  \label{fig:lm4ResidualsQQ}
  \vspace*{-3mm}
\end{figure}

Next, we check for influential observations by examining leverages, studentized residuals, and Cook's distances. From these case influence statistics there are 17 potentially problematic observations. Partial residual plots for \texttt{is_holiday} and \texttt{is_school_day} (two of the insignificant variables from the regression) are shown in Figure \ref{fig:lm4Pres}. If we ignore the 17 potentially problematic observations (coded with blue triangles in Figure \ref{fig:lm4Pres}), it appears as though holidays tend to have fewer felonies than non-holidays, and school days tend to have more felonies than non-school days.

% two-column figure
\begin{figure}[!h]
  \vspace*{-3mm}
  \centering
  \captionsetup{width=0.9\textwidth}
  \subfloat%[caption text]
  		{\includegraphics[width=0.5\textwidth]
  		{figures/lm4PresIsHoliday.png}\label{fig:lm4PresIsHoliday}}
  \hfill
  \subfloat%[caption text]
  		{\includegraphics[width=0.5\textwidth]
  		{figures/lm4PresIsSchoolDay.png}\label{fig:lm4PresIsSchoolDay}}
  \vspace*{-3mm}
  \caption{Partial residual plots for \texttt{is_holiday} and \texttt{is_school_day}. In each plot, the 17 potentially problematic observations are coded with blue triangles.}
  \label{fig:lm4Pres}
  \vspace*{-3mm}
\end{figure}




We re-fit the regression from Section \ref{sec:modelFeloniesMultipleRegression}, but now excluding the 17 potentially problematic observations. The regression summary is shown in Table \ref{tab:lm5}.

% lm5
%\begin{codeSmall}
%                           Estimate Std. Error t value   Pr(>|t|)    
%(Intercept)               202.29838    4.72876  42.780    < 2e-16 ***
%temp_min_degF               1.47704    0.07321  20.175    < 2e-16 ***
%any_precip1               -20.41161    7.20033  -2.835    0.00486 ** 
%is_holiday1               -13.26886    6.82627  -1.944    0.05275 .  
%is_school_day1              6.69136    3.12660   2.140    0.03306 *  
%day_of_week2                4.01255    4.81800   0.833    0.40553    
%day_of_week3                0.10726    4.73001   0.023    0.98192    
%day_of_week4               -0.18818    4.73899  -0.040    0.96835    
%day_of_week5               -2.77336    4.80166  -0.578    0.56393    
%day_of_week6               22.99155    4.79145   4.798 0.00000241 ***
%day_of_week7               19.50039    4.21983   4.621 0.00000545 ***
%temp_min_degF:any_precip1   0.13801    0.13747   1.004    0.31615    
%\end{codeSmall}


% lm5
% xtable(as.data.frame(summary(lm5)$coefficients), digits=c(100, 2, 2, 2, 4), display=c("s", "f", "f", "f", "g"))
\begin{table}[ht]
\vspace*{-1mm}
\footnotesize
\centering
\begin{tabular}{rrrrr}
  \hline
 & Estimate & Std. Error & t value & Pr($>|t|$) \\ 
  \hline
(Intercept) & 202.30 & 4.73 & 42.78 & $<2 \times 10^{-16}$ \\ 
  temp\_min\_degF & 1.48 & 0.07 & 20.17 & $<2 \times 10^{-16}$ \\ 
  any\_precip1 & -20.41 & 7.20 & -2.83 & 0.004863 \\ 
  is\_holiday1 & -13.27 & 6.83 & -1.94 & 0.05275 \\ 
  is\_school\_day1 & 6.69 & 3.13 & 2.14 & 0.03306 \\ 
  day\_of\_week2 & 4.01 & 4.82 & 0.83 & 0.4055 \\ 
  day\_of\_week3 & 0.11 & 4.73 & 0.02 & 0.9819 \\ 
  day\_of\_week4 & -0.19 & 4.74 & -0.04 & 0.9683 \\ 
  day\_of\_week5 & -2.77 & 4.80 & -0.58 & 0.5639 \\ 
  day\_of\_week6 & 22.99 & 4.79 & 4.80 & $2.409 \times 10^{-6}$ \\ 
  day\_of\_week7 & 19.50 & 4.22 & 4.62 & $5.448 \times 10^{-6}$ \\ 
  temp\_min\_degF:any\_precip1 & 0.14 & 0.14 & 1.00 & 0.3161 \\ 
   \hline
\end{tabular}
\captionsetup{width=0.9\textwidth}
\vspace*{-2mm}
\caption{Regression summary from the multiple linear regression of \texttt{felonies}, after excluding 17 problematic observations.}
\label{tab:lm5}   
\vspace*{-3mm}
\end{table}



%Residual standard error: 20.85 on 336 degrees of freedom
%Multiple R-squared:  0.6773,	Adjusted R-squared:  0.6667 
%F-statistic:  64.1 on 11 and 336 DF,  p-value: < 2.2e-16


After removing the 17 problematic observations, the holiday indicator and school day indicator are now significant. There is suggestive, but inconclusive evidence that \texttt{felonies} is associated with holidays --- on average, holidays have 13 fewer felonies than non-holidays (95\% CI 0 to 27 fewer felonies; two-sided p-value 0.05275). There is moderate evidence that felonies is associated with school days --- on average, school days have 7 more felonies that non-school days (95\% CI 1 to 13 more felonies; two-sided p-value 0.03306). The interaction between temperature and precipitation still isn't significant, even after removing the 17 observations (two-sided p-value 0.31615).


%If hitting the page limit, probably no need to remove the interaction term (models lm6 and lm7).




\section{Modeling the Day-to-Day Change in the Number of Felonies}
\label{sec:modelingFeloniesDiff}


Addressing the third question of interest as a supplementary analysis, we now explore if day-to-day changes in the number of felonies, as compared to the previous day, are associated with large increases in temperature after taking our other covariates into account. The theory here is that spikes in temperature (e.g., at the beginning of a heat wave or on the first few days of spring) might influence the number of felonies that occur. In this section, \texttt{temp_jump} indicates if the temperature has increased by more than $8\degf$ as compared to the previous day (see Section \ref{sec:dataset}).




\subsection{Assumptions}
\label{sec:feloniesDiffAssumptions}


In this section we use multiple linear regression to model \texttt{felonies_diff} (the difference in the number of felonies as compared to the previous day; see Section \ref{sec:dataset}). Again we assume the 4 assumptions of linear regression, as outlined in Section \ref{sec:feloniesAssumptions}.

\subsection{Statistical Analysis}
\label{sec:modelFeloniesDiffMultipleRegression}


We now perform a multiple linear regression of \texttt{felonies_diff} on the variables shown in the regression summary in Table \ref{tab:lmd2}.

% lmd2
%\begin{codeSmall}
%                          Estimate Std. Error t value Pr(>|t|)    
%(Intercept)              -19.09877    6.66217  -2.867 0.004397 ** 
%temp_jump1                21.73759   16.38144   1.327 0.185381    
%temp_min_degF              0.12321    0.09743   1.265 0.206821    
%any_precip1               -1.90533    3.72032  -0.512 0.608873    
%is_holiday1                0.95049    9.39575   0.101 0.919480    
%is_school_day1             6.95124    4.60154   1.511 0.131779    
%day_of_week2              12.22189    6.99960   1.746 0.081669 .  
%day_of_week3               2.91295    6.97539   0.418 0.676491    
%day_of_week4              11.05330    6.99797   1.580 0.115120    
%day_of_week5               9.94665    6.94328   1.433 0.152872    
%day_of_week6              25.59847    6.96531   3.675 0.000275 ***
%day_of_week7               1.79654    6.14494   0.292 0.770183    
%temp_jump1:temp_min_degF  -0.28253    0.34895  -0.810 0.418679    
%\end{codeSmall}



% lmd2
% xtable(as.data.frame(summary(lmd2)$coefficients), digits=c(100, 2, 2, 2, 4), display=c("s", "f", "f", "f", "g"))
\begin{table}[ht]
\vspace*{-1mm}
\footnotesize
\centering
\begin{tabular}{rrrrr}
  \hline
 & Estimate & Std. Error & t value & Pr($>|t|$) \\ 
  \hline
(Intercept) & -19.10 & 6.66 & -2.87 & 0.004397 \\ 
  temp\_jump1 & 21.74 & 16.38 & 1.33 & 0.1854 \\ 
  temp\_min\_degF & 0.12 & 0.10 & 1.26 & 0.2068 \\ 
  any\_precip1 & -1.91 & 3.72 & -0.51 & 0.6089 \\ 
  is\_holiday1 & 0.95 & 9.40 & 0.10 & 0.9195 \\ 
  is\_school\_day1 & 6.95 & 4.60 & 1.51 & 0.1318 \\ 
  day\_of\_week2 & 12.22 & 7.00 & 1.75 & 0.08167 \\ 
  day\_of\_week3 & 2.91 & 6.98 & 0.42 & 0.6765 \\ 
  day\_of\_week4 & 11.05 & 7.00 & 1.58 & 0.1151 \\ 
  day\_of\_week5 & 9.95 & 6.94 & 1.43 & 0.1529 \\ 
  day\_of\_week6 & 25.60 & 6.97 & 3.68 & 0.0002747 \\ 
  day\_of\_week7 & 1.80 & 6.14 & 0.29 & 0.7702 \\ 
  temp\_jump1:temp\_min\_degF & -0.28 & 0.35 & -0.81 & 0.4187 \\ 
   \hline
\end{tabular}
\captionsetup{width=0.9\textwidth}
\vspace*{-2mm}
\caption{Regression summary from the multiple linear regression of \texttt{felonies_diff}.}
\label{tab:lmd2}   
\vspace*{-3mm}
\end{table}




%Residual standard error: 31.23 on 352 degrees of freedom
%Multiple R-squared:  0.103,	Adjusted R-squared:  0.07247 
%F-statistic:  3.37 on 12 and 352 DF,  p-value: 0.0001127



After accounting for temperature, precipitation, holidays, school days, and day of week, the data provides no evidence that changes in the number of felonies is associated with large increases in temperature (two-sided p-value 0.185381). The interaction between temperature and the temperature jump indicator is also not significant (two-sided p-value 0.418679).

The data provides overwhelming evidence that \texttt{felonies_diff} is associated with \texttt{day_of_week} (p-value 0.003788 from an extra sum of squares F-test) --- on Fridays, for example, there are on average 26 more felonies than on the preceding Thursday (95\% CI 12 to 39 more felonies; two-sided p-value 0.000275).


\subsection{Model Checking and Improvement}
\label{sec:modelFeloniesDiffModelCheckingImprovement}


Neither a residual plot nor a Q-Q plot (not shown) reveal any significant violations of our assumptions. Case influence statistics reveal 12 potentially influential observations, but there is no change in interpretation once they're removed.

To simplify the model, we now iteratively remove the insignificant variables from the regression, and end up with a model in which only \texttt{temp_jump} and \texttt{day_of_week} are remaining. The regression summary is shown in Table \ref{tab:lmd6}.

% lmd6
%\begin{codeSmall}
%             Estimate Std. Error t value Pr(>|t|)    
%             Estimate Std. Error t value Pr(>|t|)    
%(Intercept)   -13.338      4.383  -3.043  0.00252 ** 
%temp_jump1      8.942      5.423   1.649  0.10006    
%day_of_week2   16.480      6.113   2.696  0.00736 ** 
%day_of_week3    6.999      6.113   1.145  0.25305    
%day_of_week4   16.152      6.116   2.641  0.00863 ** 
%day_of_week5   14.496      6.085   2.382  0.01773 *  
%day_of_week6   30.208      6.121   4.936 1.23e-06 ***
%day_of_week7    1.746      6.127   0.285  0.77589    
%\end{codeSmall}


% lmd6
% xtable(as.data.frame(summary(lmd6)$coefficients), digits=c(100, 2, 2, 2, 4), display=c("s", "f", "f", "f", "g"))
\begin{table}[ht]
\vspace*{-1mm}
\footnotesize
\centering
\begin{tabular}{rrrrr}
  \hline
 & Estimate & Std. Error & t value & Pr($>|t|$) \\ 
  \hline
(Intercept) & -13.34 & 4.38 & -3.04 & 0.002517 \\ 
  temp\_jump1 & 8.94 & 5.42 & 1.65 & 0.1001 \\ 
  day\_of\_week2 & 16.48 & 6.11 & 2.70 & 0.007358 \\ 
  day\_of\_week3 & 7.00 & 6.11 & 1.14 & 0.253 \\ 
  day\_of\_week4 & 16.15 & 6.12 & 2.64 & 0.008634 \\ 
  day\_of\_week5 & 14.50 & 6.08 & 2.38 & 0.01773 \\ 
  day\_of\_week6 & 30.21 & 6.12 & 4.94 & $1.229 \times 10^{-6}$ \\ 
  day\_of\_week7 & 1.75 & 6.13 & 0.28 & 0.7759 \\ 
   \hline
\end{tabular}
\captionsetup{width=0.9\textwidth}
\vspace*{-2mm}
\caption{Regression summary from the multiple linear regression of \texttt{felonies_diff}, after removing the insignificant variables.}
\label{tab:lmd6}   
\vspace*{-3mm}
\end{table}





%Residual standard error: 31.17 on 357 degrees of freedom
%Multiple R-squared:  0.0941,	Adjusted R-squared:  0.07634 
%F-statistic: 5.298 on 7 and 357 DF,  p-value: 8.909e-06

The temperature jump indicator still is not significant (two-sided p-value 0.10006). Only the \texttt{day_of_week} variable has a significant relationship with day-to-day changes in the number of felonies (p-value $9 \times 10^{-6}$ from an extra sum of squares F-test) --- Fridays, for example, generally have 30 more felonies than the preceding Thursday (95\% CI 18 to 42 more felonies; two-sided p-value $1 \times 10^{-6}$).


\section{Conclusions}
\label{sec:conclusions}

\subsection{Statistical Conclusions}

The data provides overwhelming evidence that felonies is associated with temperature. For every $1 \degf$ increase in temperature, there are, on average, 1.38 additional felonies per day (95\% CI 1.23 to 1.53 felonies, two sided p-value $<2\times10^{-16}$).

After accounting for temperature, there is convincing evidence that felonies is associated with precipitation. On average there are 21 fewer felonies on days with precipitation than on days without (95\% CI 4 to 38 fewer felonies; two-sided p-value 0.014990). The data also provides overwhelming evidence that felonies is associated with day of week (p-value $3 \times 10^{-6}$ from an extra sum of squares F-test).

Initially, it did not appear that the number of felonies was associated with holidays or with school days, but after removing 17 observations with high studentized residuals, these variables became significant. From the revised model, there is suggestive, but inconclusive evidence that felonies is associated with holidays --- on average, holidays have 13 fewer felonies than non-holidays (95\% CI 0 to 27 fewer felonies; two-sided p-value 0.05275). Again from the revised model, there is moderate evidence that felonies is associated with school days --- on average, school days have 7 more felonies that non-school days (95\% CI 1 to 13 more felonies; two-sided p-value 0.03306).

In both the initial and revised models, the interaction between temperature and precipitation was not significant (two-sided p-value was $>0.3$ in both models).


From a supplementary analysis, after accounting for day of week, the data provides no evidence that day-to-day changes in the number of felonies is associated with large ($>8 \degf$) increases in temperature, as compared to the previous day (two-sided p-value 0.10006).


\subsection{Scope of Inference}

As this is purely observational data, these statistical associations cannot be used to draw any causal connections. Further, any generalization of these results to cities other than NYC for time periods other than 2015 is speculative.



%\listoftodos

%\pagebreak

% To compile the bibliography: BibTeX, PDFLaTeX, Quick Build.
%\nocite{*} % This command includes all sources to be listed in the bibliography, even if uncited in the article.
%\singlespacing
%\bibliographystyle{unsrt}
%\bibliography{bib_name}

\end{document}

