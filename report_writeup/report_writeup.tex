

\documentclass[12pt,notitlepage]{article}

%\documentclass[twocolumn,secnumarabic,amssymb, nobibnotes, aps, prd]{revtex4-1}
%\documentclass[aps,preprint]{revtex4}
%\documentclass[pra,preprint]{revtex4}
%\documentclass[pra,twocolumn]{revtex4}

\usepackage[lmargin=1.in,rmargin=1.in,tmargin=1.in,bmargin=1in]{geometry}
\usepackage{setspace} %\doublespacing
\usepackage[pdftex]{graphicx}
\usepackage{latexsym}
\usepackage{amsfonts}
\usepackage{amsmath,revsymb,graphicx} 
\usepackage{natbib}
\usepackage{longtable}
\usepackage{titling}
\usepackage[
	pdfauthor={Brian Weinstein},
	pdftitle={The Association Between Felonies in NYC and Weather and Temporal Conditions},
	bookmarks=true,
	colorlinks=true,
	linkcolor=blue,
	urlcolor=blue,
	citecolor=blue,
	pdftex,
	linktocpage=true
	]{hyperref}
\usepackage[textsize=tiny]{todonotes}
\usepackage{authblk}
\usepackage{float}
\usepackage{caption}


\usepackage{xspace}


\begin{document}

\title{The Association Between Felonies in NYC and Weather and Temporal Conditions}
\author{Brian Weinstein}
\affil{\textit{Columbia University} \\ \textit{STAT W4201: Advanced Data Analysis}}
\date{May 2, 2016}

\maketitle



\begin{abstract}
\singlespacing

\noindent \textbf{Background:} The New York City Police Department recently released incident-level felony data to the New York City Open Data portal. The dataset includes timestamped information for all felonies committed in NYC.

We first examine the association between the daily number of felonies committed in NYC in 2015 and temperature, presence of precipitation, day of week, federal and New York holidays, and school days. Second, we examine the association between large increases in temperature ($>8 ^\circ$F from the previous day) and increases in the number of felonies.

\noindent \textbf{Methods and Results:} We initially test for a difference between the number of felonies on warmer and cooler days ($\geq 51.98 ^\circ$F and $< 51.98 ^\circ$F, respectively --- the NYC 2015 median), finding overwhelming evidence of a difference, with there being 44 more felonies, on average, on warmer days than on cooler day (95\% CI 38 to 51 felonies; two sided p-value $<0.000001$ from a two-sample t-test).

After accounting for presence of precipitation, holidays, school days, and day of week, the data provides overwhelming evidence that for every $1 ^\circ$F increase in temperature there are, on average, 1.4 additional felonies per day (95\% CI 1.3 to 1.6 felonies; two-sided p-value $<0.000001$ for a test that the linear regression coefficient is 0).

We next find that, after accounting for day of week, there is little evidence to suggest that increases in felonies from the previous day are associated with large increases in temperature (two-sided p-value 0.1001 for a test that the linear regression coefficient is 0).

\noindent \textbf{Conclusions:} There is a clear association between warmer temperatures and an increased number of felonies. There is no evidence that large increases in temperature from the previous day are associated with increases in the number of felonies.




\end{abstract}



\pagebreak

\singlespacing



\section{Introduction}


\subsection{describing the problem of interest}

Questions of interest:
\begin{itemize}
\item asdf
\end{itemize}


\subsection{the data set}

\subsubsection{Data sources}

\begin{itemize}
\item NYPD 7 Major Felony Incidents: data.cityofnewyork.us/d/hyij-8hr7
\item NYPD Motor Vehicle Collisions: data.cityofnewyork.us/d/h9gi-nx95
\item National Centers for Environmental Information (weather conditions and temperature data): ncdc.noaa.gov/cdo-web/search
\item New York State Holidays: cs.ny.gov/attendance\_leave/2015\_legal\_holidays.cfm
\item Federal Holidays: opm.gov/policy-data-oversight/snow-dismissal-procedures/federal-holidays/\#url=2015
\item School Attendance: github.com/ajschumacher/NYCattends/tree/master/xml
(The NYC Department of Education also publishes this data to the NYC Open Data portal, but historical data is only retained there for the current school year. Instead I’ll scrape the daily XML snapshots stored in the linked Github repository.)
\end{itemize}

\subsubsection{Data cleaning}


\subsubsection{EDA}


\subsection{the organization of the entire report}






\section{statistical model and statistical analysis}


\subsection{model setup: the statistical model employed and assumptions}


\subsection{statistical analysis / inference}






\section{model checking (and model improvement if the originally proposed model is not appropriate)}






\section{Conclusion}






\pagebreak

% To compile the bibliography: BibTeX, PDFLaTeX, Quick Build.
%\nocite{*} % This command includes all sources to be listed in the bibliography, even if uncited in the article.
%\singlespacing
%\bibliographystyle{unsrt}
%\bibliography{bib_name}

\end{document}

